\documentclass[11pt]{article}
\usepackage[margin=1in]{geometry}
\usepackage{amsmath, amssymb, amsthm}
\usepackage{algorithm, algpseudocode}
\usepackage{hyperref}
\usepackage{tcolorbox}

% Theorem environments
\newtheorem{theorem}{Theorem}[section]
\newtheorem{lemma}[theorem]{Lemma}
\theoremstyle{definition}
\newtheorem{definition}[theorem]{Definition}
\newtheorem{example}[theorem]{Example}

\title{CS3000 Algorithms - Course Notes}
\author{Your Name}
\date{Fall 2025}

\begin{document}
\maketitle
\tableofcontents
\newpage

\section{Asymptotics and Mathematical Foundations}

    \subsection{Asymptotic Notations}        
        \paragraph{Big-O Notation: $f(n) = O(g(n))$}
        There is some positive factor $k$ and a threshold value $n_0$ such that for any $n$ beyond this threshold, the value of $f(n)$ is upper bounded by $k \cdot g(n)$.
        $$\exists k > 0, \exists n_0, \forall n > n_0 \text{ such that: } |f(n)| \leq k \cdot g(n)$$
        $$\lim_{n \to \infty} \sup \frac{f(n)}{g(n)} < \infty$$
        
        \paragraph{Big-Omega Notation: $f(n) = \Omega(g(n))$}
        There is some positive factor $k$ and a threshold value $n_0$ such that for any $n$ beyond this threshold, the value of $f(n)$ is lower bounded by $k \cdot g(n)$.
        $$\exists k > 0, \exists n_0, \forall n > n_0 \text{ such that: } f(n) \geq k \cdot g(n)$$
        $$\lim_{n \to \infty} \inf \frac{f(n)}{g(n)} > 0$$
        
        \paragraph{Big-Theta Notation: $f(n) = \Theta(g(n))$}
        There are positive factors $k_1$ and $k_2$ and a threshold value $n_0$ such that for any $n$ beyond this threshold, the value of $f(n)$ is bounded both above and below by multiples of $g(n)$.
        $$\exists k_1, k_2 > 0, \exists n_0, \forall n > n_0 \text{ such that: } k_1 \cdot g(n) \leq f(n) \leq k_2 \cdot g(n)$$
        $$0 < \lim_{n \to \infty} \frac{f(n)}{g(n)} < \infty$$
        
        \paragraph{Little-o Notation: $f(n) = o(g(n))$}
        For every positive factor $k$ (no matter how small), there exists a threshold value $n_0$ such that for any $n$ beyond this threshold, the value of $f(n)$ is strictly less than $k \cdot g(n)$.
        $$\forall k > 0, \exists n_0, \forall n > n_0 \text{ such that: } |f(n)| < k \cdot g(n)$$
        $$\lim_{n \to \infty} \frac{f(n)}{g(n)} = 0$$
        
        \paragraph{Little-omega Notation: $f(n) = \omega(g(n))$}
        For every positive factor $k$ (no matter how large), there exists a threshold value $n_0$ such that for any $n$ beyond this threshold, the value of $f(n)$ is strictly greater than $k \cdot g(n)$.
        $$\forall k > 0, \exists n_0, \forall n > n_0 \text{ such that: } f(n) > k \cdot g(n)$$
        $$\lim_{n \to \infty} \frac{f(n)}{g(n)} = \infty$$
            


\end{document}